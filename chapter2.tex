\chapter{Equations and Symbols}
%------------------------------------------------

\section{Inline Math}
The Pythagorean theorem is expressed as $a^2 + b^2 = c^2$.

%------------------------------------------------
\section{Displayed Equations}

\begin{equation}
	E = mc^2
\end{equation}


%------------------------------------------------
\section{Align Environment}

\begin{align}
	f(x) &= x^2 + 3x + 2 \\
	&= (x+1)(x+2)
\end{align}


%------------------------------------------------
\section{Fractions, Roots, and Exponents}

\[
\frac{a + b}{c}, \quad \sqrt{2}, \quad x^{n+1}
\]


%------------------------------------------------
\section{Summations and Integrals}

\[
\sum_{i=1}^{n} i = \frac{n(n+1)}{2}, \quad \int_0^1 x^2\,dx = \frac{1}{3}
\]


\newpage
%------------------------------------------------
\section{Greek Symbols}

\subsection{Lowercase Greek Letters}

\[
\alpha, \ \beta, \ \gamma, \ \delta, \ \epsilon, \ \zeta, \ \eta, \ \theta, \ \iota, \ \kappa, \ \lambda, \ \mu, \ \nu, \ \xi, \ \pi, \ \rho, \ \sigma, \ \tau, \ \upsilon, \ \phi, \ \chi, \ \psi, \ \omega
\]


\subsection{Uppercase Greek Letters}

\[
\Gamma, \ \Delta, \ \Theta, \ \Lambda, \ \Xi, \ \Pi, \ \Sigma, \ \Upsilon, \ \Phi, \ \Psi, \ \Omega
\]


\subsection{Variant Greek Symbols}

\[
\varepsilon, \ \vartheta, \ \varpi, \ \varrho, \ \varsigma, \ \varphi
\]



%------------------------------------------------
\section{Matrices}

\begin{equation}
	A = 
	\begin{bmatrix}
		1 & 2 \\
		3 & 4 \\
	\end{bmatrix}
\end{equation}


%------------------------------------------------
\section{Systems of Equations}

\begin{equation}
	\begin{cases}
		x + y = 1 \\
		2x - y = 3 \\
	\end{cases}
\end{equation}


%------------------------------------------------
\section{Numbered Equations with Labels}

\begin{equation}
\label{eq:newton}
	F = ma
\end{equation}

As shown in Equation~\ref{eq:newton}, force is proportional to mass and acceleration.


%------------------------------------------------
\section{2 DOF Robot Arm Dynamic Model}

El modelo dinámico de un robot manipulador de $n$ grados de libertad está dado por la siguiente expresión:

\begin{equation}
	M(\mathbf{q})\ddot{\mathbf{q}} + C(\mathbf{q}, \dot{\mathbf{q}})\dot{\mathbf{q}} + \mathbf{g}(\mathbf{q}) = \mathbf{\tau}
\end{equation}

Específicamente para un robot manipulador de dos grados de libertad se tiene:

\begin{equation}
	\begin{bmatrix}
		M_{11}(\mathbf{q}) & M_{12}(\mathbf{q})\\
		M_{21}(\mathbf{q}) & M_{22}(\mathbf{q})\\
	\end{bmatrix}\ddot{\mathbf{q}} 
	+
	\begin{bmatrix}
		C_{11}(\mathbf{q}, \dot{\mathbf{q}}) & C_{12}(\mathbf{q}, \dot{\mathbf{q}})\\
		C_{21}(\mathbf{q}, \dot{\mathbf{q}}) & C_{22}(\mathbf{q}, \dot{\mathbf{q}})\\
	\end{bmatrix}\dot{\mathbf{q}}
	+
	\begin{bmatrix}
		g_{1}(\mathbf{q})\\
		g_{2}(\mathbf{q})\\
	\end{bmatrix}
	=
	\mathbf{\tau},
\end{equation}

Donde:

\begin{equation}
	\begin{aligned}
		M_{11}(\mathbf{q}) &= m_{1}l_{c1}^{2} + m_{2}\left[l_{1}^{2} + l_{c2}^{2} + 2l_{1}l_{c2}\cos(q_{2}) \right]  + I_{1} + I_{2}\\
		M_{12}(\mathbf{q}) &= m_{2}\left[l_{c2}^{2} + l_{1}l_{c2}\cos(q_{2})\right] + I_{2}\\
		M_{21}(\mathbf{q}) &= m_{2}\left[l_{c2}^{2} + l_{1}l_{c2}\cos(q_{2})\right] + I_{2}\\
		M_{22}(\mathbf{q}) &= m_{2}l_{c2}^{2} + I_{2}\\
		\\
		C_{11}(\mathbf{q}, \dot{\mathbf{q}}) &= -m_{2}l_{1}l_{c2}\sin(q_{2})\dot{q}_{2}\\
		C_{12}(\mathbf{q}, \dot{\mathbf{q}}) &= -m_{2}l_{1}l_{c2}\sin(q_{2})\left[\dot{q}_{1} +\dot{q}_{2}\right] \\
		C_{21}(\mathbf{q}, \dot{\mathbf{q}}) &= m_{2}l_{1}l_{c2}\sin(q_{2})\dot{q}_{1}\\
		C_{22}(\mathbf{q}, \dot{\mathbf{q}}) &= 0\\
		\\
		g_{1}(\mathbf{q}) &= \left[m_{1}l_{c1} + m_{2}l_{1}\right]g\sin(q_{1}) + m_{2}l_{c2}g\sin(q_{1}+q_{2}) \\
		g_{2}(\mathbf{q}) &= m_{2}l_{c2}g\sin(q_{1}+q_{2}).\\
	\end{aligned}
\end{equation}

Considerando a $\mathbf{q}=[q_{1}, q_{2}]^{\top}$ y $\dot{\mathbf{q}}=[\dot{q}_{1}, \dot{q}_{2}]^{\top}$ como variables de estado, el modelo dinámico del robot se puede reescribir de la siguiente forma: 

\begin{equation}
	\frac{d}{dt}
	\begin{bmatrix}
		\mathbf{q}\\
		\\
		\dot{\mathbf{q}}\\
	\end{bmatrix}
	=
	\begin{bmatrix}
		\dot{\mathbf{q}}\\
		\\
		M(\mathbf{q})^{-1}\left[\mathbf{\tau} - C(\mathbf{q}, \dot{\mathbf{q}})\dot{\mathbf{q}} - \mathbf{g}(\mathbf{q})\right] 
	\end{bmatrix}
\end{equation}