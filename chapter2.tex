\chapter{Equations and Symbols}
%------------------------------------------------

\section{Inline Math}
The Pythagorean theorem is expressed as $a^2 + b^2 = c^2$.

%------------------------------------------------
\section{Displayed Equations}

\begin{equation}
	E = mc^2
\end{equation}


%------------------------------------------------
\section{Align Environment}

\begin{align}
	f(x) &= x^2 + 3x + 2 \\
	&= (x+1)(x+2)
\end{align}


%------------------------------------------------
\section{Fractions, Roots, and Exponents}

\[
\frac{a + b}{c}, \quad \sqrt{2}, \quad x^{n+1}
\]


%------------------------------------------------
\section{Summations and Integrals}

\[
\sum_{i=1}^{n} i = \frac{n(n+1)}{2}, \quad \int_0^1 x^2\,dx = \frac{1}{3}
\]


\newpage
%------------------------------------------------
\section{Greek Symbols}

\subsection{Lowercase Greek Letters}

\[
\alpha, \ \beta, \ \gamma, \ \delta, \ \epsilon, \ \zeta, \ \eta, \ \theta, \ \iota, \ \kappa, \ \lambda, \ \mu, \ \nu, \ \xi, \ \pi, \ \rho, \ \sigma, \ \tau, \ \upsilon, \ \phi, \ \chi, \ \psi, \ \omega
\]


\subsection{Uppercase Greek Letters}

\[
\Gamma, \ \Delta, \ \Theta, \ \Lambda, \ \Xi, \ \Pi, \ \Sigma, \ \Upsilon, \ \Phi, \ \Psi, \ \Omega
\]


\subsection{Variant Greek Symbols}

\[
\varepsilon, \ \vartheta, \ \varpi, \ \varrho, \ \varsigma, \ \varphi
\]



%------------------------------------------------
\section{Matrices}

\begin{equation}
	A = 
	\begin{bmatrix}
		1 & 2 \\
		3 & 4 \\
	\end{bmatrix}
\end{equation}


%------------------------------------------------
\section{Systems of Equations}

\begin{equation}
	\begin{cases}
		x + y = 1 \\
		2x - y = 3 \\
	\end{cases}
\end{equation}


%------------------------------------------------
\section{Numbered Equations with Labels}

\begin{equation}
\label{eq:newton}
	F = ma
\end{equation}

As shown in Equation~\ref{eq:newton}, force is proportional to mass and acceleration.

