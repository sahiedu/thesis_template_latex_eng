\chapter{Figures and Tables}

\section{Figures}\index{Figure}

Supported formats include: `.png', `.jpg', `.jpeg', `.pdf', and `.eps' (depending on the compiler). It is recommended to use `.pdf' for vector images and `.png' for raster graphics.

\begin{figure}[H]
	\centering
	\includegraphics[width=0.4\linewidth]{image}
	
	\caption{A sample figure with a caption.}
	\label{fig:sample_figure}
\end{figure}

Refer to Figure~\ref{fig:sample_figure} to cite a full figure in your text.


\begin{figure}[H]
	\centering
	\subfloat[Left image\label{fig:left_img}]{
		\includegraphics[width=0.3\linewidth]{a}
	}
	\quad
	\subfloat[Right image\label{fig:right_img}]{
		\includegraphics[width=0.3\linewidth]{b}
	}
	
	\caption{Two images displayed side by side.}
	\label{fig:side_by_side}
\end{figure}

Refer to subfigures individually as Figures~\ref{fig:left_img} and~\ref{fig:right_img}, or collectively as Figure~\ref{fig:side_by_side}.


\begin{figure}[H]
	\centering
	\subfloat[Image 1 caption\label{fig:image1}]{
		\includegraphics[width=0.3\linewidth]{a}
	}
	\quad
	\subfloat[Image 2 caption\label{fig:image2}]{
		\includegraphics[width=0.3\linewidth]{b}
	}
	
	\qquad
	
	\subfloat[Image 3 caption\label{fig:image3}]{
		\includegraphics[width=0.3\linewidth]{c}
	}
	\quad
	\subfloat[Image 4 caption\label{fig:image4}]{
		\includegraphics[width=0.3\linewidth]{d}
	}
	
	\caption{Comparison of four different images.}
	\label{fig:four_images}
\end{figure}

You can reference subfigures individually, such as Figure~\ref{fig:image2}, or the full group using Figure~\ref{fig:four_images}.


%------------------------------------------------

\section{Tables}\index{Tables}

\begin{table}[H]
	\centering
	\caption{Experimental results for each treatment.}
	\begin{tabular}{c S[table-format=1.7] S[table-format=1.3]}
		\toprule
		\textbf{Treatment} & {\textbf{Response 1}} & {\textbf{Response 2}} \\
		\midrule
		Treatment 1 & 0.0003262 & 0.562 \\
		Treatment 2 & 0.0015681 & 0.910 \\
		Treatment 3 & 0.0009271 & 0.296 \\
		\bottomrule
	\end{tabular}
	\label{tab:treatment_responses}
\end{table}



\begin{table}[H]
	\caption{Sample parameters with corresponding physical units.}
	\begin{tabular*}{\linewidth}{@{\extracolsep{\fill}} c l S[table-format=2.2] l }
		\toprule
		\textbf{Symbol} & \textbf{Description} & \textbf{Value} & \textbf{Units} \\
		\midrule
		$A$ & Sample parameter A & 12.50 & \si{\meter} \\
		$B$ & Sample parameter B & 3.14 & \si{\kilogram} \\
		$C$ & Sample parameter C & -0.98 & \si{\second} \\
		\bottomrule
	\end{tabular*}
	\label{tab:sample_parameters}
\end{table}


This is an example of referencing a table using its assigned label. As shown in Table~\ref{tab:sample_parameters}.

