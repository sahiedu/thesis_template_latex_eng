\chapter{Text and Structure}

\section{Paragraphs of Text}\index{Paragraphs of Text}

\lipsum[1-1] % Dummy text

%------------------------------------------------
\section{Text Styling}
\textbf{Bold}, \textit{Italic}, \underline{Underline}, \texttt{Monospaced}, \emph{Emphasized}



%------------------------------------------------
\section{Quotations}
This is an inline quote: \enquote{This is quoted.}

\begin{quote}
	This is a block quote. It's often used for long quoted passages, interviews, or highlighted text.
\end{quote}

%------------------------------------------------
\section{Footnotes}
Footnotes are useful for adding clarifications, additional references, or brief comments without interrupting the main flow of the text.\footnote{Footnotes should be used sparingly in academic writing and should not contain essential information.}

%------------------------------------------------

\section{Hyperlinks}\index{hyperlinks}
Hyperlinks are used to reference online documentation, datasets, software repositories, or official standards. Whenever possible, links should be embedded in descriptive text rather than shown as raw URLs.

\subsection{General Example}
Comprehensive documentation and tutorials are available on the official \LaTeX\ project website
(\href{https://www.latex-project.org}{LaTeX Project Website}). In cases where the address itself is relevant, a full URL may be displayed explicitly:
\url{https://ctan.org}


\subsection{Technical Documentation Example}
We follow the installation steps provided in\awesomefootnote{https://www.latex-project.org/get/}, selecting a minimal \LaTeX{} distribution instead of a full installation.





%------------------------------------------------

\section{Lists}\index{Lists}



\subsection{Numbered List}\index{Lists!Numbered List}

Numbered lists are useful when the order of items is important or when describing step-by-step procedures.


\begin{enumerate}
	\item The first item
	\item The second item
	\begin{enumerate}
		\item Sub-item one
		\item Sub-item two
		\begin{enumerate}
			\item Sub-sub-item A
			\item Sub-sub-item B
		\end{enumerate}
	\end{enumerate}
	\item The third item
\end{enumerate}



\subsection{Bullet Points}\index{Lists!Bullet Points}

Bullet point lists are appropriate for unordered items where sequence does not matter.


\begin{itemize}
	\item The first item
	\item The second item
	\begin{itemize}
		\item Sub-item one
		\item Sub-item two
		\begin{itemize}
			\item Sub-sub-item A
			\item Sub-sub-item B
		\end{itemize}
	\end{itemize}
	\item The third item
\end{itemize}


\newpage
\subsection{Fontawesome Lists}\index{Lists!Font Awesome}

Custom symbols, such as Font Awesome icons, can be used to visually distinguish list items or emphasize specific categories.

\begin{itemize}
	\item[\footnotesize\faRobot] The first item
	\item[\footnotesize\faRobot] The second item
	\begin{itemize}
		\item[\footnotesize\faMicrochip] Sub-item one
		\item[\footnotesize\faMicrochip] Sub-item two
	\end{itemize}
	\item[\footnotesize\faRobot] The third item
	
\end{itemize}


\subsection{Descriptions and Definitions}\index{Lists!Descriptions and Definitions}

Description lists are used to associate terms or labels with brief explanations or definitions.


\begin{description}
	\item[Name] Description
	\item[Word] Definition
	\item[Comment] Elaboration
\end{description}






%------------------------------------------------

\section{Citation}\index{Citation}


\begin{itemize}
	\item Papers: \cite{Sonneveldt2007}, \cite{Snell1992}, \cite{Bugajski1992}.
	\item Books: \cite{Etkin1996}, \cite{Stevens2016}, \cite{Isidori1995}, \cite{Stengel2004}.
	\item Thesis: \cite{Harkegard2001}, \cite{Harkegard2003}.
\end{itemize}




